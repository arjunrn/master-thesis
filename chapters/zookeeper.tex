\begin{savequote}[100mm]
...ZooKeeper, a service for coordinating processes of distributed applications. Since ZooKeeper is part of critical infrastructure, ZooKeeper aims to provide a simple and high performance kernel for building more complex coordination primitives at the client.
\qauthor{---Authors of \textup{ZooKeeper: Wait-free coordination for Internet-scale systems}}
\end{savequote}

\chapter{Zookeeper}
In the simplest of terms ZooKeeper is a distributed service with the goal to assist in coordination and synchronization between its clients.

\section{ZooKeeper Architecture}

Figure ~\ref{fig:ZooKeeperArchitecture} depicts the architecture of a three server ZooKeeper cluster. There are three nodes\textit{(Server 1, Server 2, Server 3)} which accept requests from clients to store, update and delete data. These nodes cooperate with each other and coordinate the writes so that they contain identical copies of data. These nodes are called as ZooKeeper servers. When a ZooKeeper server is started it has to be provided with an ID and also the addresses of the other ZooKeeper servers. All the servers should contain an identical copy of the ID and the address information.

The servers keep their respective datastores in memory which restricts the amount of data which can be stored. The servers also keep a log of the transactions which were performed so that the data can be recovered in case of failures.

\addvspace{2.5em}
\begin{figure}[h!]
  \centering
  \begin{tikzpicture}
    %\draw[help lines,step=1] (-2,-2) grid (10,4);
    
    \foreach \x in {0,1,2,3,4,5}
      {
          \fill[TolColor3] (\x*2.5,0) rectangle(\x*2.5+2,1);
          \FPeval{\result}{clip(\x+1)}
          
          \node at (\x*2.5+1,0.5) {Client \result};
      }
      
    \fill[TolColor6!40!white] (0,2.5) rectangle(14.5,4.5);
    
    \foreach \x in {0,1,2}
    {
        \fill[TolColor4] (\x*5+0.5,3) rectangle(\x*5+0.5+3.5,4);
        \FPeval{\result}{clip(\x+1)}
        \node at (\x*5+2,3.5) {Server \result};
    }
    
    \draw[->>, thin] (1,1)  to [out=90,in=-90,looseness=1] (2.25,3);
    \draw[->>, thin] (3.5,1)  to [out=90,in=-90,looseness=1] (2.25,3);

    \draw[->>, thin] (6,1)  to [out=90,in=-90,looseness=1] (7.5,3);
    \draw[->>, thin] (8.5,1)  to [out=90,in=-90,looseness=1] (7.5,3);
    \draw[->>, thin] (11,1)  to [out=90,in=-90,looseness=1] (7.5,3);
    
    \draw[->>, thin] (13.5,1)  to [out=90,in=-90,looseness=1] (12.25,3);
    
  \end{tikzpicture}
  \caption{ZooKeeper Architecture}
\end{figure}

\addvspace{1.0em}

The ZooKeeper servers communicate with each other using a consensus protocol called ZAB~\cite{junqueira2011zab} which is modeled on Paxos but is not entirely similar. When a request is received on the servers the request is forwarded to the leader which computes the necessary changes and then broadcasts it to the rest of the servers. When a majority of the servers respond with an acknowledgment the transaction can be committed and a response can be sent to the client. 

The size of the quorum is the factor against which performance is measured. Three is the smallest number of servers required for a quorum. One failure is tolerated but both the other two servers must be always available. This setup generally gives the fastest throughput. A quorum needs three or more odd number of servers. Five servers is the next largest and seven and nine come after that. Quorums larger than these are generally not used in production.

For requests which do not modify the state of the server like data read requests, there is no need for consensus. All the servers contain identical copies of the data and any read request can be processed locally on the server. 

The clients are initialized with the addresses of all or some of the addresses of the servers in the ZooKeeper cluster. The client then chooses one of the servers and tries to connect to it through a TCP connection. If the connection cannot be made or the connection breaks during operation then the client attempts to connect to another server.

\subsection{Client Sessions}
When a ZooKeeper client connects to a ZooKeeper cluster this creates a session associated with that instance of the connection. Whenever the client loses the connection to the server it attempts to reconnect to another server in the cluster. If it cannot reconnect within a specified amount of time then the session associated with that client connection is deleted. When the client subsequently connects at some point in the future then a new session is created. Any locks which are held by a client are deleted when the session is lost. The server periodically sends a pulse to the client to check if it is still alive. If it does not receive an acknowledgment it resends the pulse again. Finally after a preconfigured timeout the client is considered to be lost and the session is deleted.

\subsection{ZooKeeper Data Organization}
ZooKeeper data is presented like a UNIX file system. There are no files and directories, only \textit{znodes}. Every znode can have data associated with it and also have children. Other meta-data about the znode like the creation timestamp, the owner of the znode, the version of the data and any access-control information is also stored along with the data. Figure ~\ref{fig:ZKDataOrganization} shows a part of tree typical in ZooKeeper cluster. The circle at the top represents the root node and always has the path \textit{/}. In this example the root node has two child nodes viz. \textit{/app1} and \textit{/app2}. The znode \textit{/app1} in turn has 3 children called \textit{p\_1}, \textit{p\_2} and \textit{p\_3}. These are leaf nodes and are represented as hexagons. The full path for znode \textit{p\_1} would be \textit{/app1/p\_1}.

ZooKeeper has two types of znodes: \textit{regular} and \textit{ephemeral} znodes. Ephemeral znodes are present only as long at the client which created them is connected to the cluster. If the connection drops or the client crashes causing the connection to drop the the ephemeral znode is deleted. Ephemeral znodes also are different because they cannot have child znodes. Attempting to create a child znode for an ephemeral znode throws an exception. Regular znodes are persisted across sessions. There are also sequential znodes which have a counter appended to their names. All sequential nodes which are siblings and have the same name have an incrementing value appended to the name as sequential znodes are created. Znodes can also be both sequential and ephemeral in which case the znodes have a counter value appended to the end of names and they are deleted when the client which created them is disconnected.
\vspace{1.5em}
\begin{figure}[h!]
  \centering
  \begin{tikzpicture}[node distance=1.5cm]
    \filldraw[fill=TolColor3, draw=black, thick] (4,5) circle (15pt);
    \filldraw[fill=TolColor3, draw=black, thick] (2,2.5) circle (15pt);
    \filldraw[fill=TolColor3, draw=black, thick] (6,2.5) circle (15pt);
    \node[thick, regular polygon, regular polygon sides=6, minimum width=30pt,draw, rotate=30, fill=TolColor4] (reg1) at (0,0){};
    \node[thick, regular polygon, regular polygon sides=6, minimum width=30pt,draw, rotate=30, fill=TolColor4] (reg1) at (2,0){};
    \node[thick, regular polygon, regular polygon sides=6, minimum width=30pt,draw, rotate=30, fill=TolColor4] (reg1) at (4,0){};

    \draw (0,0.55) -- (2,2);
    \draw (2,0.55) -- (2,2);
    \draw (4,0.55) -- (2,2);
    \draw (2,3) -- (4,4.5);
    \draw (6,3) -- (4,4.5);
    \draw[densely dashed] (6,0) -- (6,2);

    \node at (3,5) {/};
    \node at (0.9,2.5) {/app1};
    \node at (7.2,2.5) {/app2};

    \node at (0,-1) {/app1/p\_1};
    \node at (2,-1) {/app1/p\_1};
    \node at (4,-1) {/app1/p\_1};

  \end{tikzpicture}
  \centering
  \caption{ZooKeeper Data Organization}
\end{figure}


\subsection{ZooKeeper Guarantees}
ZooKeeper has two ordering guarantees:
\begin{enumerate}
	\item \textbf{Linearizable Writes:} All the update operations on the cluster can be serialized and they respect precedence.
	\item \textbf{Client FIFO Order:} All operations from a client are executed in the order they were sent from the client.
\end{enumerate}
ZooKeeper also has the following \textbf{Liveness} and \textbf{Durability} guarantees: As long as a quorum of servers are available the service can accept, process and commit requests. Any previously committed changes will be always present as long as a quorum of servers are available.

\subsection{Cluster Reconfiguration}
	The current version of ZooKeeper cannot be reconfigured dynamically. To add more servers to a cluster the cluster has to be first stopped. The new servers have to be added to the configuration information of the cluster members and then finally the cluster has to be restarted. This is a manual process which is prone to errors and can cause down-time due to misconfiguration or split brain. This entails downtime. But it is also important that new servers be added dynamically to balance load and to replaces servers which have crashed. By utilizing some of the properties of the ZooKeeper system a reconfiguration method~\cite{shraer2012dynamic} was proposed which can rebalance the clients dynamically and migrate the data without shutting down the cluster.

\subsection{Why not a Distributed Database?}
At a first glance, it may appear that a Distributed Database like Cassandra or BigTable, HBase may solve this coordination problem. However these databases are actually built using a distributed co-ordination service like ZooKeeper. ZooKeeper which is maintained by the Apache Foundation is used as a component in Cassandra. Similarly Chubby which was developed as a co-ordination service by Google is used in GFS and BigTable.

\section{Kazoo Library}
Since the connections are made through TCP there are several languages which have libraries to communicate with ZooKeeper. The Python~\cite{van2002python} library which is provided with the ZooKeeper package is quite small and contains limited features. Kazoo is a Python library which contains many features some of which are borrowed from Netflix's Curator~\cite{apachecurator2015}. It is a pure Python implementation which means that it doesn't have any external C dependencies. It has support for older versions of ZooKeeper viz. 3.4 and 3.3. It has also support for evented IO through gevent. It also provides many higher order primitives like Watchers, Locks, Barriers. For these reason Kazoo is used as the base for the ParKazoo library. ParKazoo tries to maintain the same API as Kazoo. It can be in fact for most applications used as a drop-in replacement for Kazoo.

\subsection{Library API}
The Kazoo library provides the following functions. Some of them are identical to the functions in the Python library which is bundled along with ZooKeeper. They are:
  \subsubsection{Initialization of Kazoo}
  	\begin{lstlisting}
  		def __init__(hosts, timeout=10.0, randomize_hosts=True)
  	\end{lstlisting}
  	To initialize a \textit{Kazoo} client object we need a list of server addresses to which the client can connect. This is provided as a Python \textit{list} of \textit{string}s. The \textit{timeout} argument specifies how long before the client should throw an \textit{Exception} when it cannot connect to the servers. The \textit{randomize\_hosts} parameter indicates if a random server from the list should be selected or the first server should be chosen.
	The client object also provides the following functionality.
	\begin{enumerate}
		\item Connectivity Checks
		\item Client State Notifications
		\item State Listeners
	\end{enumerate}

  \subsubsection{Sync Operations}
  	\begin{lstlisting}
  		def sync(path)
  	\end{lstlisting}
  	If a client wants to perform a read operation on a znode but it wants all the pending operations on that znode to be synchronized first, it calls the \textit{sync} operation on that path. A sync is like a partial update operation. It does not require a consensus between the servers. But it must be sent to the leader and wait till all operation on that path are processed and committed. 

  \subsubsection{Server Commands}
  	\begin{lstlisting}
  		def command(cmd='ruok')
  	\end{lstlisting}
  	ZooKeeper servers provide administration commands which can be used to obtain information about the configuration, the statistics of the performance of the cluster and to reset the statistics. There are also commands to list the currently connected clients and the watches present on the znodes. These commands are can be run from the client by using the \textit{command()} operation.

  \subsubsection{Create Node}
  \begin{lstlisting}
    def create(path, value, ephemeral=True, sequential=True, make_path=False, watch=None)
  \end{lstlisting}
  This call creates a znode in the tree structure. The \textit{path} parameter indicates the path of the node, the \textit{value} parameter should be a byte array containing the data to be stored. \textit{make\_path} indicates if the the ancestor znodes should also be created if they do not yet exist. The \textit{watch} parameter leaves a watch on the node which invokes a callback function when the node is modified. The type of change event and the corresponding new data is passed as an argument to the callback function. The \textit{sequential} and \textit{ephemeral} flags indicate if the znode to be created should be of those corresponding types.
  
  \subsubsection{Read a Node}
  \begin{lstlisting}
    def get(path, watch=None)
  \end{lstlisting}
  Fetching the data is possible when the znode is present in the data tree. If the znode does not exist then an \textit{Exception} indicating this is thrown. Also the \textit{watch} argument can be used to leave a watch on the znode which is triggered when the znode is modified.
  
  \subsubsection{Deleting a Node}
  \begin{lstlisting}
    def delete(path, recursive=False)
  \end{lstlisting}
  If the node has children then the znode can be deleted by setting the \textit{recursive} argument to \textit{True}. Otherwise the delete call throws an exception which indicates that the znode is not empty.
  
  \subsubsection{Getting Children of a Node}
  \begin{lstlisting}
    def get_children(path, watch=None, include_data=False)
  \end{lstlisting}
  This call is used to retrieve all the the children of a znode. It can also set a watch which notifies the client when new znodes are created below the znode in question. By default, this call only returns the names of the child nodes. If the data of the child znodes is also required then the \textit{include\_data} parameter should be set to \textit{True}.
  
  These are only some of the operations which are provided by the Kazoo library. The ones listed above are used as primitives in the construction of the ParKazoo library. There are also corresponding asynchronous versions for each of the above. The difference between the synchronous and asynchronous versions of ZooKeeper operations is that the client blocks until the result of the operation is returned in case of synchronous operations. In the case of synchronous operations the function call instead returns a deferred object which can be used to notify the application when the result is ready. The is mostly used in single threaded evented programs.
  
\subsection{Library Recipes}
  ZooKeeper was designed to be simple and does not directly implement complex synchronization and coordination mechanisms generally required by most applications. It does not provide higher order constructs like locks, queues or counters. However these lower order primitives can be used to construct such functions. The Kazoo library includes in itself some of the higher level constructs which can be used by the application programmer. Some of them are listed below. There constructs are also later implemented in the ParKazoo library without much change from their original Kazoo implementations.
  
  \subsubsection{Barrier}
    Barriers are synchronization mechanisms for concurrent threads of execution. Multiple ZooKeeper clients which have a common barrier will all block on the barrier until a certain condition is met and the barrier is removed programmatically Double barriers can be used to synchronized the beginning and the end of a distributed asynchronous task. Double barriers differ from regular barrier because they wait for a predetermined number of clients to arrive at the barrier. When the number of clients is reached all of them cross the barrier together. When the barrier section is complete the same process occurs when the clients are leaving the barrier.
  
  \subsubsection{Counter}
  This is a race-free counter which allows multiple nodes to share a counter value. A Counter type object can be used directly with the plus and minus operators in the Python application. However only integer and float values are permitted.
  
  \subsubsection{Election}
  When multiple clients want to perform a distributed task which requires one of the to be a leader/primary they can use the Election construct. ZooKeeper will then elect one of the participating clients as the leader and this client will be notified through a callback function which was passed to the \textit{Election} object when it was created. This construct also provides the functionality to query for all the client participating in the contest to be the leader.
  
  \subsubsection{Locks}
    Locks are the most commonly used synchronization mechanism in concurrent programs. Only one client can acquire the lock and hold it at any point of time. ZooKeeper does not provide mandatory locks. Advisory locks are implemented in the Kazoo library A lock can be created by a client on a specific znode. Subsequently any number of clients can compete to acquire the lock. If the lock cannot be acquired then the client blocks till the lock is acquired. The acquirement of the lock can be also canceled at any point before acquisition. Also the list of all the contenders for the lock can be obtained.
    
    Kazoo also provides ~\textit{Semaphore} objects which are similar to the Python Semaphores in the ~\textit{threading} module~\cite{pythonSemaphore}. Semaphore are similar to locks but also allow multiple client to hold the Semaphore. The Kazoo Semaphore object can be initialized with an initial count which indicates the number of available resources. Whenever one of the clients acquires the semaphore the count is decremented. This way the number of clients accessing the distributed resource can be regulated. Once a client is done with a Semaphore then it is released. Only as many number of clients which will fit into the Semaphore are then notified. This prevents the ZooKeeper servers from being overwhelmed with requests to acquire the Semaphore.
  
  \subsubsection{Partitioner}
  This construct is used to divide the members of a set among the members of a party, such that every member receives zero or more items of and each item is only given to one member. And example of this would be a task queue. Here the tasks ares the items of the set and the clients of ZooKeeper are the party members which receive the tasks.
  
  \subsubsection{Party}
    Parties are used to keep a registry of nodes. Client nodes may enter or leave a Party which updates the membership registry of the party.
  
  \subsubsection{Queue}
    Kazoo supports simple \textit{Queue}s and a improved implementation called ~\textit{LockingQueue} which provides locking and priority support. The simple Queue can be used to add items into the queue. The consumers of the queue can remove items from the queue. However if the the consumer client which acquired an item from the Queue crashes then the queue item is lost even though it might not have been processed. The LockingQueue provides items to consumers but does not remove the item from the queue. It creates an ephemeral node corresponding to the item which indicates that the item is being processed. In case the consumer client crashes then the ephemeral znode is deleted and another consumer client can acquire the item. Once the consumer finishes processing the item it should explicitly inform the LockingQueue that it has processed the queue item at which time the queue removes the item from the queue.
  
  \subsubsection{Watchers}
  A client may need to watch for only certain types of changes to a znode, such as changes to the data of the znode or its children. The watches left on the znodes by the client calls like \textit{get}, \textit{get\_children} or \textit{set} are triggered whenever the znode is updated irrespective of the type of event. The watchers provide more fine grained control. They can also be disabled by returning a False Boolean function from the callback function.
  
  \subsection{Transactions}
  ZooKeeper has the facility to perform atomic transactions which consist of a sequence of operations. A transaction allows a client to perform a sequence of operations on the cluster such that the operations and their results are not interleaved with operations from other clients. From the client's perspective it appears like the sequence of operations is processed as a single operation on the ZooKeeper cluster. This is a common feature on many databases and can be a useful feature in constructing more complex functionality. An application which uses the Kazoo client can request a \textit{Transaction} object from the Kazoo client. Then it performs \textit{get}, \textit{create}, \textit{update} and \textit{delete} operations within the scope of this transaction. Then finally it calls \textit{commit()} on the \textit{Transaction} object. This sends the transaction to the clusters which performs the operations while upholding the constraints of the transaction. If the transaction can be completed then the Kazoo client informs the application by returning \textit{True} to the application. If the transaction fails then it is automatically retried. If after a fixed number of retries the transaction cannot be still completed then a \textit{False} is returned to the application.
