\chapter{Limitations}

\section{Data Inconsistency}
Not all operations are completely data safe. The recursive data operation is not completely safe. The delete() operation iterates over the client connections and then deletes the subtrees. Assume that the client crashes after deleting the subtrees only a few clusters. Then if one of the nodes which is missing then the whole tree gets recreated. So a node which is supposed to be deleted still exists.


\section{Ordering of Operations}
ZooKeeper provides the guarantee of client operation order. This means that all the operations which are issues by a client are executed in the order they are sent to the server. But with ParKazoo this assumptions will not hold true. Assume 2 operations issued by a client. The first operation is mapped to one cluster and the second operation gets mapped to another cluster. If the operations are issued one after the other in a normal ZooKeeper operation the result of the first operation is returned first. But with ParKazoo the result of the first operation could be returned first because the operations are executed on different clusters.
