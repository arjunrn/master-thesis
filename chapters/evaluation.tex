\chapter{Evaluation}

The peak write throughput performance of the original ZooKeeper and Parkazoo are compared.

\section{Test Setup}
The size of the quorum is the factor against which performance is measured. Three is the smallest number of servers required for a quorum. One failure is tolerated but both the other servers must be available. This setup generally gives the fastest throughput. A quorum needs 3 or more odd number of servers. Five servers is the next largest and seven and nine come after that. Quorums larger than these are generally not used in production. Our test setup consists of comparing the performance of a three server ZooKeeper cluster against a nine server ParKazoo ensemble.

Multiple nodes make requests to the cluster or the ensemeble. Each node has multiple processes and each process has in turn multiple threads. Each thread has its own Kazoo or ParKazoo client object to make requests to the servers.
Kazoo/ParKazoo provides a recipe for barrier which is used by the clients to start making requests. All the nodes start making request when they clear the barrier. Every request is recorded with a timestamp and the time it took to complete the request. All these timestamped requests from each thread are collected and finally they are written to the disk. At the end of the run all the files containing the requests timestamp and duration are collected. They are then aggregated. They are bucketed by the to the second in which they were executed. This gives the number of requests from all the test nodes that occurred in a second. The average request rate and their latency is recorded. 


As we can see from the throughput rates when the number of clients is smaller the throughput of the ParKazoo is almost 20\%-30\% higher than the equivalent ZooKeeper setup. However when the number of ParKazoo clients on a single node increases the throughput rate fails to grow and at a certain point drops below the throughput rate of the the equilvalent ZooKeeper test node. But the total throughput of ZooKeeper plateaus out at around 6300 requests/seconds. However the request throughput of ParKazoo can reach almost 8000 requests/second.

The reason for the deteriorated performance of ParKazoo when more processes and threads are present is probably due to the higher number of open connections. This issue may neeed to be investigated further. 
