\chapter{Related Work}
Camelot~\cite{hastings1990distributed} is early system which proposed a locking service using transactions in a distributed environment. 
There have been several projects which implement a distributed coordination service similar to ZooKeeper. The basis for such systems are consensus algorithms. Paxos~\cite{lamport2001paxos} is the most popular consensus algorithm and modified versions of it have been implemented in Chubby~\cite{burrows2006chubby} and other services. State Machine Replication~\cite{schneider1990implementing} is also an important concept used in the implementation of ZooKeeper.

Boxwood~\cite{maccormick2004boxwood} presents a distributed storage medium which also uses Paxos at its core.

dynamoDB~\cite{vogels2012amazon}

In 2010 Flavio Juqueira, one of the creators of ZooKeeper proposal a partitioned implementation~\cite{junqueira2010partitioned} of ZooKeeper. This proposal is similar to our implementation However it also proposes another component on the main ensemble which routes the requests from the clients to the destination clusters. The distribution of the znodes can also be dynamic, which means that the client which creates a znode can specify at runtime which cluster the nodes gets mapped to. This complicates the use of the system because the application developer needs to keep compute and map nodes to individual clusters. This proposal also relies on the assumption that the requests are always distributed uniformly across the tree. In case of hotspot the tree has to be split and spread across multiple ensembles.

There was also an attempt to modify the ZooKeeper core~\cite{biligiri2014multiquorum} itself with multiple quorums. Instead of single leader in a quorum, every server has the chance to be the leader for a subset of the data. The author's argument is that this would lead to more uniform utilization of resources in case of hotspots. 

