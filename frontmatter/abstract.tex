With the advent of cloud computing it has become very common for large applications to be built using distributed platforms. While distributed systems offer a cost advantage they are create new challenges. Coordination and synchronization mechanisms which are common on single computers need to be reinvented for a distributed environment. Most distributed applications require a central store for coordination tasks like synchronization, configurations and task queues. Coordination services are used for this purpose and should have the property of being performant, fault-tolerant and scalable. ZooKeeper is distributed co-ordination service which is used a many distributed applications. It is based on the ZAB protocol which ensures consistency and total-ordering of the data updates on the data. However this protocol is limited by overhead for quorum between the participant nodes. As the size cluster grows larger the overhead increases. But to increase the throughput of the cluster should be able to be increased by adding more nodes. We attempt to solve the quorum overhead problem by partitioning the data namespace. Multiple sets of ZooKeeper clusters are used so that each one is reponsible for a part of the complete namespace.  This way the load of the data is partitioned between clusters and performance of the ensemsble can scale as the load increases.
